% Chapter Template

\chapter{Introduction} % Main chapter title

\label{Chapter1} % Change X to a consecutive number; for referencing this chapter elsewhere, use \ref{ChapterX}

%----------------------------------------------------------------------------------------
%	SECTION 1
%----------------------------------------------------------------------------------------
Taking a photograph of an object, traditionally, we need to face a camara (detector) to the object. But with Two-photon imaging 
we use a detector that is towards the light source, rather than towards the object.
As the name suggest it, Two-photon Imaging, we also use another photon that is strongly correlated.
 

Two-photon imaging has been demonstrated using two types of light sources. Type-one
two-photon imaging uses entangled photon pairs as the light source. In 1995 Pittman, realized a 
quantum two-photon geometric optical effect.  They have successfully performed optical 
imaging by means of a quantum-mechanical entangled source\cite{pittman}.

Type-two of imaging uses chaotic light. The type-two  
image-forming correlation is caused by the superposition between paired two-photon amplitudes,
or the symmetrized effective two-photon wave-function\cite{physicsGhost}.


\subsection{Computational Two-photon Imaging}

When we see the the different setups shown here, (Figures \ref{fig:twoPhotonSetup}, \ref{fig:pittman} and \ref{fig:thermalSetup}),
it can be noted that the transmitted path of the light, consists of a light beam propagating through air. What diferences these setup is the nature of the light used.
Specially in the setup using thermal light, the transmitted path is a classical phenomenon that can be simulated by a computer using Fresnel's propagation theory.
In Figure \ref{computationalSetup} we can see the setup for a Computational
Two-photon Imaging, where the transmitted path of the light is replaced by a computational simulation generated by a computer.
%image

The Computational Two-photon Imaging allows us to simulate the electric field data to be obtained before, during or after the data from the reflected arm is generated,
eliminating the need for collecting the data generated in the transmitted path. Also we have to use less opto-electronical elements on the optical table, simplifying 
the original setup and reducing considerably the amount of data generated. The resulting detection module consist only in one detector, a bucket detector that collects
 a single pixel (no spatial information) on light which has been transmitted through or reflected from the object.
In this situation only one light beam and one photodetector are required, this means that this imaging configuration cannot depend
on non-local two-photon interference.

Event though this realisation of the Two-photon Imaging is relevant in this discussion, 
because it is posible to retrive the image by precomputing the intensity fluctuation pattern that 
would have been seen by a high-spatial-resolution detector in a lensless two-photon imaging. 
It means we introduced some kind of coherence in this single path two-photon imaging. This is done by putting a CW
laser beam through a spatial light modulator (SLM) whose inputs are chosen to create the desired coherence behavior\cite{computational}.
A SLM is a special device that can manipulate light by modulating the amplitude, phase or polarization of the light waves in the two dimensions of space and time.
%----------------------------------------------------------------------------------------
%	SECTION 2
%----------------------------------------------------------------------------------------



