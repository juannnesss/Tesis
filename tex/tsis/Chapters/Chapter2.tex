% Chapter Template

\chapter{Theory} % Main chapter title

\label{Chapter2} % Change X to a consecutive number; for referencing this chapter elsewhere, use \ref{ChapterX}

In Here we will discus some important facts to get a complete understanding 
in the physical phenomena that is happening. Specially we will develop the 
notions that are crucial in the understanding of the Two-photon imaging using entangled light, been this said 
we will start talking about correlations.

\section{Correlations between two photons}

The term "correlation" is crucial at this point, and it refers to the relation of two or more situations have. For example 
we can establish a correlations between the US dolar currency exchange rate and the prices of technology in one country. These two things have direct relation, if one blows up, the other one will too.
These two situations, or variables, can have a strong correlations or a week one. \\

Indeed in quantum physics we can have a pair of photons that are so strongly correlated, in their possible variables (spatial and temporal),
that we say they are entangled. This statment can leads us 
to a dense discussion about the nature of this entanglement, 
a discussion that were started between Einstein and Bohr in the first years of quantum physics \cite{einstein}.\\

To avoid this discussion we will just talk about correlations, 
and when referring about a pair of correlated photons, we will mean that 
this pair of photons are correlated in one or varius of their variables. 
They can be correlated in momentum, meaning that when one photon have a given $\vec{q}_i$ momentum 
and the other photon have a $\vec{q}_j$ momentum that is determined by the first, this relations is the momentum correlations a we can work out an expression for this relationship.
\subsection{SPDC}

As the title of this work implies, we need a source light that produces pair of photons, 
and we would like to exploit the advantages of strong correlations between them.
The photons generated via spontaneous parametric down conversion (SPDC) are
widely used in quantum optics experiments. The popularity of this source of paired
photons is strongly related to the relative simplicity of its experimental
realisation, and to the variety of quantum features that down converted photons can exhibit. 
The generated photons via SPDC can be correlated in different degrees of freedom, for example 
in polarisation, in frequency and in the equivalent degrees of freedom: 
'orbital angular momentum, space and transverse momentum \cite{spatiocorrelations}.\\

SPDC is an optical process in which focus a beam pump, that is 
propagating in the $z$-direction, to a nonlinear crystal of length $L$. Using first order perturbation 
and the paraxial approximation, the two-photon state is given by:
\begin{equation}
\label{eq:stateFunComplex}
\ket{\Psi}=\int dq_B dq_A d\Omega_B d\Omega_A 
\text{x} [\Phi(q_B,\Omega_B;q_A,\Omega_A) \hat{a}^{\dagger} (\Omega_B,q_B) \hat{a}^{\dagger}(\Omega_A,q_A) \\
+ \Phi(q_A,\Omega_A;q_B,\Omega_B) \hat{a}^{\dagger}(\Omega_B,q_B) \hat{a}^{\dagger}(\Omega_A,q_A)]   \ket{0}  
\end{equation}
Where this state function depends on the transverse wave vectors $q_n=(q_n^x,q_n^y)$ and frequency detuning, $\Omega_n=\omega_n-\omega_0^n$, 
around the central frequencies, $\omega_0^n$, for the photon at the path $A$ or $B$ ($n=A,B$).
The $\Phi(q_B,\Omega_B;q_A,\Omega_A)$ and $\Phi(q_A,\Omega_A;q_B,\Omega_B)$
are the mode functions or biphotons that contains all the informations about the correlations
between the pair of down-converted photons. The operator $\hat{a}^{\dagger}$ indicates the creations of an $n$-polarized photon with transverse momentum $q_n$, 
and frequency detuning $\Omega_n$ \cite{physicsGhost}. \\

In the optical table we put a polariser at certain directions at the detections modules,
filtering some of the photons before reaching the detector, this filtering also have a mathematical effect in our model, 
it is posible now to write \ref{eq:stateFunComplex} different, dropping one term:
\begin{equation}
\label{eq:stateFun}
\ket{\Psi}=\int dq_B dq_A d\Omega_B d\Omega_A 
\text{x} [\Phi(q_B,\Omega_B;q_A,\Omega_A) \hat{a}^{\dagger} (\Omega_B,q_B) \hat{a}^{\dagger}(\Omega_A,q_A) 
] \ket{0}  
\end{equation}

The mode function  $\Phi(q_B,\Omega_B;q_A,\Omega_A)$ is related with the joint probability of detecting both an $B$-polarized
photon, with tranverse momentum $q_B$ and frequency detuning $\Omega_B$, at the detector $B$ 
and an $A$-polarized
photon, with tranverse momentum $q_A$ and frequency detuning $\Omega_A$, at the detector $A$. 

\subsubsection{Phase matching conditions}
In particular, $\Phi(q_B,\Omega_B;q_A,\Omega_A)$ reads \cite{spatiocorrelations}:
\begin{equation}
\label{eq:mode}
\Phi(q_B,\Omega_B;q_A,\Omega_A) = \mathcal{N} \alpha(\Delta_0,\Delta_1) \beta(\Omega_B,\Omega_A) \text{ x }
sinc \left( \frac{\Delta_k L}{2} \right) e^{i \frac{\Delta_k L}{2}}
\end{equation}
Where $\mathcal{N}$ is a normalisation constant, $\alpha(\Delta_0,\Delta_1)$
and $\beta(\Omega_B,\Omega_A)$yields the informations of the pump's transverse 
and spectral distribution, respectively, L is the length of the nonlinear crystal.
For the process that is happening inside the crystal, there are some conditions that have to be fulfilled. These conditions are related with the energy and momentum conservations inside the parametric down conversion process.
The terms $\Delta_0$, $\Delta_1$ and $\Delta_k$ are functions that result from the phase matching conditions and read:
\begin{equation}
\Delta_0=q_B^x + q_A^x
\end{equation}
\begin{equation}
\Delta_1= q_A^y cos\phi_A + q_B^y cos\phi_B - N_B \Omega_B sin\phi_B + N_A \Omega_A sin\phi_A - \rho_B q_B^x sin\phi_B 
\end{equation}
\begin{equation}
\Delta_k=N_p(\Omega_B+\Omega_A)-N_B\Omega_B cos\phi_B - N_A\Omega_A cos\phi_A -q_B^y sin\Omega_B + q_A^y sin\Omega_A + \rho_p \Delta_0 - \rho_B q_B^x cos\phi_B
\end{equation}
The angles $\phi_B$ and $\phi_A$ are the creation angles of the down-
converted photons inside the crystal with respect to the pump’s
propagation direction, whereas the angles $\rho_p$ and $\rho_B$ account for
the walk-off of the pump $p$ and the $B$ down-
converted photon, respectively. 
In this study, $\phi_B$ and $\phi_A$ are treated as constants, 
mainly because the scanned transverse momentum regions represent a small portion around
the emission angles. $N_n$ denotes the inverse of the group velocity for each photon.



\subsection{Spatial Correlations}

In order to observe the correlations presented in \ref{eq:mode} we have to take into account some considerations about the descrption of the things we have in optical table.
First of all we have a pump beam with a Gaussian profile with waist $w_p$ 
in such way that $\alpha (\Delta_0,\Delta_1 ) \propto \text{exp}[-w_p^2 (\Delta_0^2 + \Delta_1^2 )/4]$, a CW pump laser, mathematically represented by
$\beta (\Omega_B , \Omega_A) \propto \delta(\Omega_B + \Omega_A)$. Making the aproximations for the sinc function by a Gaussian fuctions with the same width at $1/e^2$ of its maximum,
i.e., $sinc(x) \approx \text{exp}(-\gamma x^2)$ with $\gamma$ equal $0.193$.
The mode function reduces to:

\begin{equation}
\label{eq:modeSim}
\Phi(q_B,\Omega_B;q_A,\Omega_A) = \mathcal{N} \beta (\Omega_B , \Omega_A)
\text{ x exp}\left[ -\frac{w_p^2 (\Delta_0^2 + \Delta_1^2 )}{4}-\gamma \left(\frac{\Delta_k L}{2} \right)^2 + i\frac{\Delta_k L}{2} \right]  
\end{equation}

In order to observe the transverse correlations (spatial correlations), the 
frequency information has to be traced out, in the optical table this can
be achieved by placing some interferometer filters before detection. This spectral
filters are modeled as $f_n (\Omega_n)=\text{exp}[-\Omega_n^2/(4\sigma_n^2)]$, with bandwidth $\sigma_n$ chosen to achive a regimen where the 
spatial-spectral correlations are completely broken \cite{broke}. 
To achieve this mathematically we have to integrate \ref{eq:modeSim} around the
spatial variables:
\begin{equation}
\tilde{\Phi}(q_B,q_A) = \int d\Omega_B d\Omega_A f_B(\Omega_B)f_A(\Omega_A) \Phi(q_B,\Omega_B;q_A,\Omega_A)
\end{equation}

\subsection{Tunable Spatial Correlation SPDC source light}

\section{Imaging}

\subsection{Standar Imaging}

The observed image is a reproduction of the illuminated object, mathematically
corresponding to a convolution between the object distribution fuction $ |T(\vec{\rho_o})|^2$ (aperture function) and a $\delta$-function, which is present for the perfect
point-to-point correspondence\cite{introquantumoptics}:
\begin{equation}
I(\vec{\rho_i})=\int_{obj} d\vec{\rho_o} |T(\vec{\rho_o})|^2 \delta(\vec{\rho_o}+\frac{\vec{\rho_i}}{m})
\end{equation}
where $I(\vec{\rho_i})$ is the intensity at the image plane, $\vec{\rho_o}$ and $\vec{\rho_i}$ are 2-D vectors of the
transverse coordinates in the object and image planes, respectively, and
$m=s_i/s_o$ is the image magnification factor.

In reality, we are limited by the finite size of the optical system, we may never obtain a perfect image.
we have to take into account the constructive-destructive interference present
in this phenomena, because of the wave nature of light. The point-to-point correspondence turns into a point-to-"spot" relationship.
For further informations about this "real life" situation check the \ref{appendix:intensity}.



\subsection{Two-photon Imaging}


The optical imaging used the photons at the image plane, to form the image. In other 
words it takes measure one photon per spot at the image plane. For the type-one and type-two
two-photon imaging, in certain aspects the behaviour is similar as that of the classical.
They both exhibit a similar point-to-point imaging-forming function, except the 
two-photon image is only reproducible in the joint-detection between two independent photodetectors,
and the point-to-point imaging-forming function is in the form of second-order correlation,
\begin{equation}
R_{12}(\vec{\rho_i})=\int_{obj} d\vec{\rho_o} |T(\vec{\rho_o})|^2 G^{(2)}(\vec{\rho_o},\vec{\rho_i})
\end{equation}
where $R_{12}(\vec{\rho_i})$ is the joint-detection counting rate between photodetectors $D_1$ and $D_2$.
$G^{(2)}(\vec{\rho_o},\vec{\rho_i})$ is a nontrivial point-to-point second-order correlation
function, corresponding to the probability of observing a joint photo-detection event
at the coordinates $\vec{\rho_o}$ and $\vec{\rho_i}$. The physics behing $G^{(2)}(\vec{\rho_o},\vec{\rho_i})$
is what changes between type-one and type-two two-photon imaging.

%----------------------------------------------------------------------------------------
%	SECTION 1
%----------------------------------------------------------------------------------------
\subsubsection{Two-photon Imaging using entangled photon}

\textit{Light Source}

main source of information \cite{physicsGhost}
%-----------------------------------
%	SUBSECTION 1
%-----------------------------------
\textit{Biphoton}





\begin{equation}
\ket{\Psi}=\int dq_s dq_i d\Omega_s d\Omega_i 
\text{x} [\Phi(q_s,\Omega_s;q_i,\Omega_i) \hat{a}^{\dagger} (\Omega_s,q_s) \hat{a}^{\dagger}(\Omega_i,q_i) \\
+ \Phi(q_i,\Omega_i;q_s,\Omega_s) \hat{a}^{\dagger}(\Omega_s,q_s) \hat{a}^{\dagger}(\Omega_i,q_i)]   \ket{0}  
\end{equation}

taken like it appears on \cite{spatiocorrelations}

 Where $\Phi(q_s,\Omega_s;q_i,\Omega_i)$ are the mode fuctions or Biphotons, a fuctions that contain all the information about the correlations. $ \hat{a}^{\dagger}(\Omega_n,q_n)$ the creation of a photon with tranverse momentum $q_n$ and frequency $\Omega_n$

%-----------------------------------
%	SUBSECTION 2
%-----------------------------------

\textit{Mode Function}


\begin{equation}
\Phi(q_s,\Omega_s;q_i,\Omega_i) \propto E_p(q_p,\Delta_0) B_p(\Omega_p) \mathcal{C}_{spatial}(q_s) \mathcal{C}_{spatial}(q_i) 
 x \mathcal{F}_{frequency}(\Omega_s) \mathcal{F}_{frequency}(\Omega_i) sinc \left( \frac{\Delta_k \mathcal{L}}{2} \right)
\end{equation}
where $B_p(\omega^0_p+\Omega_p)$ and $E_p(q_p)$ are the frequency and transverse momentum distribution of the pump. $\mathcal{C}_{spatial}(q_n)$ spatial filtering. $\mathcal{F}_{frequency}(\Omega_n)$ frequency filter function.



\textit{Gaussian approximations}

\cite{spatiocorrelations}

Taking into account the Gaussian nature of the pump, that's $E_p(q^x_p , q^y_p ) \approx exp \left[ -\frac{w_p^2}{4}(q^{x^2}_p + q^{y^2}_p )\right]$.

approximating the sinc function by a Gaussian function with the same width at $\frac{1}{e^2}$ of its maximum, i.e., $sinc(x)\approx exp(- \gamma x^2)$  with $\gamma$ equal 0.193. 

\begin{equation}
\mathcal{F}_{frequency}(\Omega_n) \approx exp \left[-\frac{ \Omega^2_n}{4 \sigma^2_n} \right] 
\end{equation}
\begin{equation}
\tilde{\Phi}(q_s,q_i)=\int d\Omega_s d\Omega_i \mathcal{F}_s (\Omega_s) \mathcal{F}_i (\Omega_i) \Phi(q_s,\Omega_s;q_i,\Omega_i)
\end{equation}

The Biphoton then takes a quadratic form:
\begin{equation}\label{eq:quadratic}
\tilde{\Phi}(q_s,q_i)=N exp\left[ -\frac{1}{2}x^T A x + i b^T x \right]
\end{equation}
where N is a normalization constant, $x$ is a 4-dimensional vector defined as $x = (q^x_s, q^y_s ,q^x_i,q^y_i )$, $A$ is a 4 x 4 real-valued, symmetric, positive definite matrix and b is a 4- dimensional vector. A and b are defined from the phase-matching conditions of the SPDC process. $x^T$ and $b^T$ denote the transpose of $x$ and $b$. $A$ and $b$ are functions that depend of all the relevant parameters in the experiment such as the length of the crystal L, pump waist $w_p$, creation angles inside the crystal $\varphi_n$ and the width of the spectral filter $\sigma_n$.

A way to quantify the degree of spatial correlation we shall define 'correlation parameter':
\begin{equation}
K^\lambda = \frac{C^\lambda_{si}}{\sqrt{C^\lambda_{ss}C^\lambda_{ii}}}
\end{equation}
calculated for each direction $(\lambda = x, y)$ from the covariance matrix $C^\lambda$ with elements $C^\lambda_{kj} = \langle q^\lambda_k q^\lambda_j \rangle - \langle q^\lambda_k \rangle \langle q^\lambda_j \rangle $.


\textit{Fresnel Propagator}


Fresnel Propagator: $h(r,z)=(- \frac{i}{\lambda z})e^{(i \frac{2 \pi z}{\lambda})} \Psi (r,z)$ 
with $\Psi(r,z) = e^{(i \frac{\pi}{\lambda z })r^2}$. Thin-lens transfer function $L_f (r)=\Psi(r,-f)$
 \\
\begin{equation}\label{eq:green}
G= \int d^2 r_1 \int d^2 r_0 h(r_f - r_1,f) L_f(r_1) h(r_1 - r_0,f)
\end{equation}

The propagation is done by determining the Green function\cite{green} of the optical path
by which the beam will travel. The biphoton function
in terms of transverse momenta $\Phi_1 (q_s , q_i )$ after traveling
through two arbitrary optical paths can be expressed
in terms of the corresponding Green functions and the
initial biphoton function $\Phi(q_s , q_i )$ as:

\begin{equation}
\Phi_1 (q_s , q_i )= G_s(q_s,r_1) G_i(q_i,r_2) \Phi (q_s,q_i) 
\end{equation}
\begin{equation}
\Phi_1 (r_1 , r_2 )= \int d^2 q_s d^2 q_i \Phi_1 (q_s , q_i ) 
\end{equation}

Taking advantage of the 2-F system as a Fourier-Transform to reduce the amount of calculations. Solving \ref{eq:green} over $r_0$ and $r_1$ we have:
\begin{equation}
G(q,r_f)=C e^{\frac{i \pi}{\lambda f} r_f^2} e^{\frac{i \lambda f}{4 \pi} q^2} \delta ( q - \frac{2 \pi}{\lambda f}r_f)
\end{equation}
where C is a complex constant that depends only on $\lambda = 2\pi c$ and $f$. Then we can define the Green Functions for each path:

\begin{equation}
G_1(q_s,r_1)=G(q_s,r_1) x T(r_1) 
\end{equation}
\begin{equation}
G_2(q_i,r_2)=G(q_i,r_2)
\end{equation}

Where $T(r_1)$ is the transfer function of the object.

Gathering all the previous results we can obtain $\Phi_1 (r_1 , r_2 )=C^2 T(r_1) \Phi (\frac{2 \pi}{\lambda f}r_1, \frac{2 \pi}{\lambda f}r_2)$, which describes the biphoton at the planes of the object and the scanning detector. It shows that the biphoton at the 2F plane in terms of
$r_1$ and $r_2$  has the same form as the biphoton at the
output face of the crystal with the relationship $q = \frac{2 \pi}{\lambda f} r$.
This allows to computationally simulate the biphoton at the 2-F plane by using Eq \ref{eq:quadratic} without the need to computationally simulate its propagation through the 2-F system.

We are collecting all the light that interacts with the object by the means of a bucket detector, this from the mathematical point of view leave us with: 
 $\Phi_1 (r_2) = C^2 \int d^2 r_1 T(r_1) \Phi (\frac{2 \pi}{\lambda f}r_1, \frac{2 \pi}{\lambda f}r_2)$ 
The coincidence counts that will be measured by the Detectors will be proportional to the magnitude square of the resulting biphoton function $\Phi_1 (r_2)$.
\begin{equation}
S(r_2) \propto |  \int d^2 r_1 T(r_1) \Phi (\frac{2 \pi}{\lambda f}r_1, \frac{2 \pi}{\lambda f}r_2) |^2
\end{equation}

For non-ideal forms of $\Phi (q_s,q_i)$ we have the relation between $\Phi (q) \rightarrow \Phi (r)$ for a 2F system, Hence: $\Phi(r)=\frac{1}{\sqrt{det(\Sigma)(2 \pi)^4}} e^{- \frac{1}{2} r^T \Sigma^{-1} r} e^{ibr}$ \\
$\Sigma=$


--------------------------------------------------
\subsubsection{Two-photon Imaging Using Chaotic Sources}