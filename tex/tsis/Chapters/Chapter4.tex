% Chapter Template

\chapter{Results} % Main chapter title

\label{Chapter4} % Change X to a consecutive number; for referencing this chapter elsewhere, use \ref{ChapterX}

%----------------------------------------------------------------------------------------
%	SECTION 1
%----------------------------------------------------------------------------------------
In This chapter we present the experimental data recovered through the differents 
steps described in the Chapter before. Most the following results consists in 
2-D arrays representing a matrix, where in each position a color is painted, depending
on how many photons were detected in single or coincidences counts.
As we have seen, before making a Two-photon image, there are some process 
that have to be made before. The first thing todo is to achieve a Gaussian behavior 
of the original diode laser. For doing do we have to look at the beam propagation after 
the Spatial Filter. 

\section{Achiving a Gaussian Beam}

PYTHON PROGRAM STILL ON PROGRESS, TALK ABOUT M FACTOR

\section{Finding The Correlated Photons}

After obtaining a Gaussian propagation, and achieve a pump waist that no varies to much
while propagates, we focused the laser at the BBO crystal and with the help of the \textit{waist lens}
we set the $w_p=91 \mu m$. Before observe the spatial correlations of the down converted
photons, we make sure we are seing them. Figure \ref{fig:correlatedPhotonSpot} shows 
two different images recovered, where in Fig.\ref{fig:correlatedPhotonSpot}(A) we found
out that the translational mount of the mask was not well placed, it was cutting some of the light.

\begin{figure}[h!]
\centering
{  \includegraphics[width=0.45\textwidth]{Figures/correlatedPhotonSpot1.png} }
{  \includegraphics[width=0.45\textwidth]{Figures/correlatedPhotonSpot2.png} }
\caption{(A) and (B) shows the B-photon of the down converted pair. In (B) we moved away the translational mount of the mask}
 \label{fig:correlatedPhotonSpot}
\end{figure}

In Figure \ref{fig:correlatedPhotonSpot}(B) we can the B-photon, and now there is
no interference by the translational mount. As said before we placed a pair of polarisers
in order to filter them. In the figure we can see that just one direction of the 
ring in \ref{fig:bbo}(b) while the other is partially filtered.


\section{Experimental Correlations }

Afterwars we would like to observe the shape of the spatial correlations the pair of down converted
photons and see the experimental behaviour of $\tilde{\Phi}(\vec{q}_B,\vec{q}_A)$.
When remembering the definition of $\vec{q}_n$, it is a 2-D vector, containing the information
of the photon in x and y direction. Since we have two photons, each one with two spatial 
variables. Resulting that we can 4 differents correlations for a pair of photons. There
is important to point out that this transverse momentum $\vec{q}_n$ is related with
his equivalent $\vec{r}_n$ the position of the photon, with n making reference to the A and B paths.

In Figure \ref{fig:expCorrelations} there are the correlations in the xx and yy  direction.
The 2-D matrix in Figure \ref{fig:expCorrelations}(\textit{XX Correlation}) is
 the result of repeating this recipe: placing the $D_A$ at a fixed position and scanning the $D_B$ just in the 
x direction, next we move de $D_A$ one position in the x direction. Repeating this N times 
we construct an image of the coincidence counts between $D_A$ and $D_B$ in every position.

\begin{figure}[h!]
\centering
{  \includegraphics[width=0.45\textwidth]{Figures/xxCorrelation.png} }
{  \includegraphics[width=0.45\textwidth]{Figures/yyCorrelation.png} }
\caption{Experimental Spatial correlations between a pair of down-converted photons. \textit{XX Correlation} shows the correlation in the x variables. \textit{YY Correlation} shows the correlation in the y variables, Beam propagating in the z direction. $w_p = 91\mu m$}
 \label{fig:expCorrelations}
\end{figure}

Figure \ref{fig:expCorrelations}(\textit{YY Correlation}) show the spatial correlation in 
the yy direction, this image is done by repeating the same recipe as before, but this 
time scanning and moving in the y direction.
The spatial correlations in this case present a negative behaviour in both XX and YY direction, an anticorrelation.
Meaning that is expected to measure a photon at a negative position at the B 
path if we measured a photon at a positive position at the A path. They both exhibit a elliptical
shape, but the YY correlation is a narrower one, meaning there is a stronger relation
between the pair of photons in the Y direction.

\section{Mask Alignment}

Before making a Two-photon image we need to know that we have placed the mask in the correct 
spot. This correct spot is defined by the Figures \ref{fig:mask1} and \ref{fig:masks}. Which
localisation was decided in function of where the flux of correlated photon was greater.
The following images where produced as the standar image is done. It means we show the shadow
of the aperture in the B path. Every position of the images is the single 
counts of the $D_B$ in the exposure time. It is like the standar image in the sense 
that we are using the spatial information of the light that interacted with the mask.

\subsection{Mask 1}
The Following Figure \ref{fig:localizationSq} shows the final localisation of the first
mask used. It was the final position because its position is really similar to the one described
in Figure \ref{fig:mask1}. For making this image we set the step length to be $0.2mm$ and
the exposure time was 1 second per position.

\begin{figure}[h!]
\centering
\includegraphics[width=0.6\textwidth]{Figures/localizationSq.png} 
\caption{Localization of the mask with an square}
\label{fig:localizationSq}
\end{figure}

\subsection{Mask 2}
Figure \ref{fig:localizationL} shows the localisation of the second mask. While in Figure \ref{fig:localizationL}(A)
there is the initial position of the mask, Figure \ref{fig:localizationL}(B) shows
the new localisation of the aperture after a translation in the y direction. This images
where done by setting the steps to $0.2mm$ and the exposure time to 1 second per position.


\begin{figure}[h!]
\centering
{  \includegraphics[width=0.45\textwidth]{Figures/localizationL1.png} }
{  \includegraphics[width=0.45\textwidth]{Figures/localizationL2.png} }
\caption{Moving the L Mask in order to put it in the most central spot}
 \label{fig:localizationL}
\end{figure}
In Figure \ref{fig:localizationDef} is presented the definitive position of the aperture
before making a Two-photon imaging. If we take a closer look to the Figure, we can appreciate
a higer contrast, this is because in this opportunity we set the steps to be $0.1mm$ and 
the exposure time to be 30 seconds per position. This image is the result of measuring for around 
14 hours.
\begin{figure}[h!]
\centering
\includegraphics[width=0.6\textwidth]{Figures/localizationLLong.png} 
\caption{Long exposure of the definitive localization of the mask, in this try we leave the 
detector in each place for 30 seconds, we also make the steps of the detector smaller, $0.1mm$}
\label{fig:localizationDef}
\end{figure}
\subsection{Mask 3}

The definitive localisation of the third mask used is presented in the Figure \ref{fig:localizationInte}, 
where the step was $0.1mm$ and the exposure time was set to 30 seconds per position.

\begin{figure}[h!]
\centering
\includegraphics[width=0.6\textwidth]{Figures/interrogationLocation.png} 
\caption{Interrogation definitive position}
\label{fig:localizationInte}
\end{figure}


\section{Two-Photon Images}

Finally we get to observe the Two-photon images that are the core of this monograh,
It is important to remember the way this images are obtained. The image $R(\vec{r}_A)$ is a 
function of the coincidence counts between $D_C$ and $D_A$. We scan $D_A$, and as a 
result we obtain a 2-D matrix where each ($i,j$) position is the coincidence count.
\subsection{mask1}
Figure \ref{fig:twoPhotonSq} is the Two-photon image of the square aperture. The maximun
coincidence counts changed position in the image, compared to Figure \ref{fig:localizationSq}.
Nevertheless the shape is identificable, still has a shape of square, but its position
is reflected in both x and y direction. 
\begin{figure}[h!]
\centering
\includegraphics[width=0.6\textwidth]{Figures/two-photonImageSq.png} 
\caption{Experimental Two-photon image recovered for the square aperture}
\label{fig:twoPhotonSq}
\end{figure}

\subsection{mask2}
The Two-photon image of the second mask is presented in Figure \ref{fig:twoPhotonL}. 
In this opportunity it is clear that the more complex shape of the aperture is hardly
identifiable. However, there are some other interesting things to note about the image.
As the original aperture, the image is not symmetrical, and it is not pointing to the
original direction the L was in Figure \ref{fig:localizationDef}. This make us to think 
about reflexions in the image respect from the original mask, but in this case is 
not that easy to detect them.
\begin{figure}[h!]
\centering
\includegraphics[width=0.6\textwidth]{Figures/twoPhoL3.png} 
\caption{Two-photon image recovered for the L aperture.}
\label{fig:twoPhotonL}
\end{figure}



\subsection{mask3}
The third Two-photon image is in Figure \ref{fig:twoPhotonInte}. Again the complex original
shape is barely visible in the image obtained, Nevertheless the image have a more rounded
part at the biger Y position, that hint us about a reflexion of the interrogation symbol,
now it is oriented like \textit{'?'}, so it may present a reflexion in both x and y direction. 
\begin{figure}[h!]
\centering
\includegraphics[width=0.6\textwidth]{Figures/twoPhotonInte.png} 
\caption{Two-Photon Image Interrogation}
\label{fig:twoPhotonInte}
\end{figure}

