% Chapter Template

\chapter{Discussions and Conclusion} % Main chapter title

\label{Chapter5} % Change X to a consecutive number; for referencing this chapter elsewhere, use \ref{ChapterX}

%----------------------------------------------------------------------------------------
%	SECTION 1
%----------------------------------------------------------------------------------------

We used a light source that generates pair of entangled photons, however, we broke this entanglement and traced out all the temporal
information. Still we were able to recover some of the characteristics of the objects used, this is an evidence of the capabilities of this source light.
SPDC is an simple experimental process with not too many technical problems that is capable of producing pair of photons
that can exhibit a strong correlation.

We presented an alternative techniche to the problem presented at the introduction of this monograh. If we lose the ability to recover
the spatial information of the reflected or scattered light from an object, we described along this work why Two-photon
imaging can be an alternative. We can light the object with an strongly correlated light, and mesure coincidences at a separated detector, and 
we may recover the image.

The Imaging process shared characteristics in both standar and Two-photon imaging vertions. As discussed through this work and in Appendix \ref{AppendixThermal},
The image recovered in coincidence counts, is a convolution between the transfer function $T(\vec{r}_o)$ of the object, 
and a Second order correlations function $G^{(2)}(\vec{r}_B,\vec{r}_A)$. Meanwhile, in the standar imaging process we find out
that the image recovered is a convolution between $T(\vec{r}_o)$ and Sombrero-like function, which acts like 
the 'correlation' function between the light at the object and at the image.




The results presented so far are not completely exempt, the numerical simulations and the experimental data have points in common. 
In the numeral simulations the images suffered reflections about the axis in wich the correlations was negative. In the experimental Two-photon
imaging of the Mask 1, Fig. \ref{fig:twoPhotonSq}, the image was flipped in both directions, this was the expected behaviour that is present
in the numerical solution.

It is clear that in order to achieve a higher spatial resolution we need to move to a regime where the spatial correlations are
stronger, meaning that they are elipsis more narrower, similar to straight lines. Wanting this narrow spatial correlations is equivalent 
to have a Sombrero-like function with a narrower point-spread function in the frame of standar imaging.

This Experimental procedure described through this monograph is the sum of many previous work, all the things done before
were necessary to get to this point were the experiment is. It is a shame the time stipulated for a monograph is so short,
the results presented so far, are just the first ones. To make an complete study of the 
phenomena of Two-photon using tuneable Spatial Correlations it is necessary to change the pump waist, in order to change the shape of 
spatial correlations.


1. signo correlacion relation with the position of the image.... how norrow a spatial correlation and the quality of the resulting image.

4. what is the true nature of the ability of recotrsucting an image, why correlation light intensities.\cite{zhong}
5. esta es la suma de varios trabajaos, se presento los resultados, primeros steps para obversavar los efectos en el ghost 


