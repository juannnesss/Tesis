% Chapter Template

\chapter{Discussions and Conclusion} % Main chapter title

\label{Chapter5} % Change X to a consecutive number; for referencing this chapter elsewhere, use \ref{ChapterX}

%----------------------------------------------------------------------------------------
%	SECTION 1
%----------------------------------------------------------------------------------------

In summary, we may conclude that ghost imaging is the result of quantum in- terference. Either type-one or type-two, ghost imaging is characterized by a non-factorizable point-to-point image-forming correlation which is caused by constructive-destructive interferences involving the nonlocal superposition of two-photon amplitudes, a nonclassical entity corresponding to different yet indistinguishable alternative ways of producing a joint photo-detection event. The interference happens within a pair of photons and at two spatially separated coordinates. The multi-photon interference nature of ghost imaging determines its peculiar features: (1) it is non- local; (2) its imaging resolution differs from that of classical; and (3) the type-two ghost image is turbulence-free. Taking advantage of its quantum interference nature, a ghost imaging system may turn a local “bucket” sensor into a nonlocal imaging camera with classically unachievable imaging resolution. For instance, using the Sun as light source for type-two ghost imaging, we may achieve an imaging spatial resolution equivalent to that of a classical imaging system with a lens of 92-meter diameter when taking pictures at 10 kilometers.10 Furthermore, any phase disturbance in the optical path has no influence on the ghost image. To achieve these features the realization of multi-photon interference is necessary\cite{physicsGhost}.