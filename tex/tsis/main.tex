%%%%%%%%%%%%%%%%%%%%%%%%%%%%%%%%%%%%%%%%%
% Masters/Doctoral Thesis 
% LaTeX Template
% Version 2.5 (27/8/17)
%
% This template was downloaded from:
% http://www.LaTeXTemplates.com
%
% Version 2.x major modifications by:
% Vel (vel@latextemplates.com)
%
% This template is based on a template by:
% Steve Gunn (http://users.ecs.soton.ac.uk/srg/softwaretools/document/templates/)
% Sunil Patel (http://www.sunilpatel.co.uk/thesis-template/)
%
% Template license:
% CC BY-NC-SA 3.0 (http://creativecommons.org/licenses/by-nc-sa/3.0/)
%
%%%%%%%%%%%%%%%%%%%%%%%%%%%%%%%%%%%%%%%%%

%----------------------------------------------------------------------------------------
%	PACKAGES AND OTHER DOCUMENT CONFIGURATIONS
%----------------------------------------------------------------------------------------

\documentclass[
12pt, % The default document font size, options: 10pt, 11pt, 12pt
oneside, % Two side (alternating margins) for binding by default, uncomment to switch to one side
english, % ngerman for German
onehalfspacing, %singlespacing, % Single line spacing, alternatives: onehalfspacing or doublespacing
%draft, % Uncomment to enable draft mode (no pictures, no links, overfull hboxes indicated)
%nolistspacing, % If the document is onehalfspacing or doublespacing, uncomment this to set spacing in lists to single
%liststotoc, % Uncomment to add the list of figures/tables/etc to the table of contents
%toctotoc, % Uncomment to add the main table of contents to the table of contents
%parskip, % Uncomment to add space between paragraphs
%nohyperref, % Uncomment to not load the hyperref package
headsepline, % Uncomment to get a line under the header
%chapterinoneline, % Uncomment to place the chapter title next to the number on one line
%consistentlayout, % Uncomment to change the layout of the declaration, abstract and acknowledgements pages to match the default layout
]{Structural} % The class file specifying the document structure
\setcounter{secnumdepth}{4} %numbering 

\usepackage[toc,page]{appendix}

\usepackage[utf8]{inputenc} % Required for inputting international characters
\usepackage[T1]{fontenc} % Output font encoding for international characters
\usepackage{physics} %for physics simbols as kets 
\usepackage{mathpazo} % Use the Palatino font by default
\usepackage{float}
\usepackage{amsmath}
%\usepackage[backend=bibtex,style=authoryear,natbib=true]{biblatex} % Use the bibtex backend with the authoryear citation style (which resembles APA)
%\usepackage{biblatex} 
%\bibliography{Bibliography}%referencias
%\addbibresource{Bibliography.bib} % The filename of the bibliography
\usepackage[nottoc,notlot,notlof]{tocbibind}
\usepackage[autostyle=true]{csquotes} % Required to generate language-dependent quotes in the bibliography
\usepackage{verbatim} 
%----------------------------------------------------------------------------------------
%	MARGIN SETTINGS
%----------------------------------------------------------------------------------------

\geometry{
	paper=a4paper, % Change to letterpaper for US letter
	inner=2.5cm, % Inner margin
	outer=3.8cm, % Outer margin
	bindingoffset=.5cm, % Binding offset
	top=1.5cm, % Top margin
	bottom=1.5cm, % Bottom margin
	%showframe, % Uncomment to show how the type block is set on the page
}

%----------------------------------------------------------------------------------------
%	THESIS INFORMATION
%----------------------------------------------------------------------------------------

\thesistitle{Two-Photon Imaging Using Tunable Spatial Correlations} % Your thesis title, this is used in the title and abstract, print it elsewhere with \ttitle
\supervisor{Dr. Alejandra \textsc{Valencia}} % Your supervisor's name, this is used in the title page, print it elsewhere with \supname
%\supervisor{Dr. Mayerlin}
\examiner{} % Your examiner's name, this is not currently used anywhere in the template, print it elsewhere with \examname
\degree{Physicist} % Your degree name, this is used in the title page and abstract, print it elsewhere with \degreename
\author{Juan \textsc{Vargas}} % Your name, this is used in the title page and abstract, print it elsewhere with \authorname
\addresses{} % Your address, this is not currently used anywhere in the template, print it elsewhere with \addressname

\subject{Physics Sciences} % Your subject area, this is not currently used anywhere in the template, print it elsewhere with \subjectname
\keywords{} % Keywords for your thesis, this is not currently used anywhere in the template, print it elsewhere with \keywordnames
\university{\href{https://uniandes.edu.co}{Universidad de Los Andes}} % Your university's name and URL, this is used in the title page and abstract, print it elsewhere with \univname
\department{\href{https://fisica.uniandes.edu.co/en/}{Physics Department}} % Your department's name and URL, this is used in the title page and abstract, print it elsewhere with \deptname
\group{\href{https://opticacuantica.uniandes.edu.co/index.php/en/home}{Quantum Optics}} % Your research group's name and URL, this is used in the title page, print it elsewhere with \groupname
\faculty{\href{https://ciencias.uniandes.edu.co}{Science Faculty}} % Your faculty's name and URL, this is used in the title page and abstract, print it elsewhere with \facname

\AtBeginDocument{
\hypersetup{pdftitle=\ttitle} % Set the PDF's title to your title
\hypersetup{pdfauthor=\authorname} % Set the PDF's author to your name
\hypersetup{pdfkeywords=\keywordnames} % Set the PDF's keywords to your keywords
}

\begin{document}

\frontmatter % Use roman page numbering style (i, ii, iii, iv...) for the pre-content pages

\pagestyle{plain} % Default to the plain heading style until the thesis style is called for the body content

%----------------------------------------------------------------------------------------
%	TITLE PAGE
%----------------------------------------------------------------------------------------

\begin{titlepage}
\begin{center}

\vspace*{.06\textheight}
{\scshape\LARGE \univname\par}\vspace{1.5cm} % University name
\textsc{\Large Thesis}\\[0.5cm] % Thesis type

\HRule \\[0.3cm] % Horizontal line
{\huge \bfseries \ttitle\par}\vspace{0.3cm} % Thesis title
\HRule \\[1.4cm] % Horizontal line
 
\begin{minipage}[t]{0.3\textwidth}
\begin{flushleft} \large
\emph{Author:}\\
{\authorname} % Author name - remove the \href bracket to remove the link
\end{flushleft}
\end{minipage}
\begin{minipage}[t]{0.3\textwidth}
\begin{flushright} \large
\emph{Supervisor:} \\
{\supname} % Supervisor name - remove the \href bracket to remove the link  
\end{flushright}
\end{minipage}\\[2cm]
 
%\vfill

\large \textit{A thesis submitted in fulfillment of the requirements\\ for the degree of \degreename}\\[0.3cm] % University requirement text
\textit{in the}\\[0.3cm]
\groupname\\\deptname\\[2cm] % Research group name and department name
 
%\vfill
\includegraphics[width=0.075\textwidth]{Logo.png} % University/department logo - uncomment to place it
 
{\large \today}\\[2.7cm] % Date

\vfill
\end{center}
\end{titlepage}

%----------------------------------------------------------------------------------------
%	DECLARATION PAGE
%----------------------------------------------------------------------------------------

\begin{declaration}
\addchaptertocentry{\authorshipname} % Add the declaration to the table of contents
\noindent I, \authorname, declare that this thesis titled, \enquote{\ttitle} and the work presented in it are my own. I confirm that:

\begin{itemize} 
\item This work was done wholly or mainly while in candidature for a research degree at this University.
\item Where any part of this thesis has previously been submitted for a degree or any other qualification at this University or any other institution, this has been clearly stated.
\item Where I have consulted the published work of others, this is always clearly attributed.
\item Where I have quoted from the work of others, the source is always given. With the exception of such quotations, this thesis is entirely my own work.
\item I have acknowledged all main sources of help.
\item Where the thesis is based on work done by myself jointly with others, I have made clear exactly what was done by others and what I have contributed myself.\\
\end{itemize}
 
\noindent Signed:\\
\rule[0.5em]{25em}{0.5pt} % This prints a line for the signature
 
\noindent Date:\\
\rule[0.5em]{25em}{0.5pt} % This prints a line to write the date
\end{declaration}

\cleardoublepage

%----------------------------------------------------------------------------------------
%	QUOTATION PAGE
%----------------------------------------------------------------------------------------

\vspace*{0.2\textheight}

\noindent\enquote{\itshape Nonesenses... later due}\bigbreak

\hfill N.N

%----------------------------------------------------------------------------------------
%	ABSTRACT PAGE
%----------------------------------------------------------------------------------------

\begin{abstract}
\addchaptertocentry{\abstractname} % Add the abstract to the table of contents
Two-photon imaging is a technique for obtaining an image of an object by means of the coincidence counts of two spatially separated detectors. Traditionally we need to face a single camera (detector) to the object we would like to take an image from, but with this technique we obtain the image by measuring the correlations between light beams.  The first experiment for Two-photon imaging was reported by Pittman et al.\cite{pittman}, where they used a source of entangled photons to perform the experiment. This pair of strongly correlated photon are generated from spontaneous parametric down-conversion (SPDC), a nonlinear optical process in which a laser pump incident on a crystal leads to the emission of a pair of photons in an entangled state. 

In our experiment we used a diode laser, centered at 405 nm, to pump a BBO crystal to obtain pairs of entangled photons (signal and idler). We make the signal photon interact with masks of different shapes and collect it at the Fourier plane. Meanwhile we scan the transverse plane of propagation of the idler photon, we count every time we get a coincidence in a given position, forming the image of the used mask. 

Using the work presented in \cite{omar} we learnt that we can tune the form of the spatial correlation between the pair of photons coming out from the BBO crystal, we control this by changing the waist pump of the laser. This work consists in measuring these different spatial correlations and checking the effect in the obtained image, how its quality and position changes with different spatial correlations. The Two-photon imaging have some peculiar features: it is non-local; its imaging resolution differs from that of classical. These features may turn a local ‘bucket’ sensor into a nonlocal imaging camera with classically unachievable imaging resolution \cite{physicsGhost}.

Lo que haremos nosotoros
Se han hecho un par de experimentos 
Vestajas nuestro experimento


\end{abstract}

%----------------------------------------------------------------------------------------
%	ACKNOWLEDGEMENTS
%----------------------------------------------------------------------------------------

\begin{acknowledgements}
\addchaptertocentry{\acknowledgementname} % Add the acknowledgements to the table of contents
The acknowledgments and the people to thank go here, don't forget to include your project advisor\ldots
s 

\end{acknowledgements}

%----------------------------------------------------------------------------------------
%	LIST OF CONTENTS/FIGURES/TABLES PAGES
%----------------------------------------------------------------------------------------

\tableofcontents % Prints the main table of contents

\listoffigures % Prints the list of figures

%\listoftables % Prints the list of tables

%----------------------------------------------------------------------------------------
%	ABBREVIATIONS
%----------------------------------------------------------------------------------------

%\begin{abbreviations}{ll} % Include a list of abbreviations (a table of two columns)

%\textbf{LAH} & \textbf{L}ist \textbf{A}bbreviations \textbf{H}ere\\
%\textbf{WSF} & \textbf{W}hat (it) \textbf{S}tands \textbf{F}or\\

%\end{abbreviations}

%----------------------------------------------------------------------------------------
%	PHYSICAL CONSTANTS/OTHER DEFINITIONS
%----------------------------------------------------------------------------------------

%\begin{constants}{lr@{${}={}$}l} % The list of physical constants is a three column table

% The \SI{}{} command is provided by the siunitx package, see its documentation for instructions on how to use it

%Speed of Light & $c_{0}$ & \SI{2.99792458e8}{\meter\per\second} (exact)\\
%Constant Name & $Symbol$ & $Constant Value$ with units\\

%\end{constants}

%----------------------------------------------------------------------------------------
%	SYMBOLS
%----------------------------------------------------------------------------------------

%\begin{symbols}{lll} % Include a list of Symbols (a three column table)

%$a$ & distance & \si{\meter} \\
%$P$ & power & \si{\watt} (\si{\joule\per\second}) \\
%Symbol & Name & Unit \\

%\addlinespace % Gap to separate the Roman symbols from the Greek

%$\omega$ & angular frequency & \si{\radian} \\

%\end{symbols}

%----------------------------------------------------------------------------------------
%	DEDICATION
%----------------------------------------------------------------------------------------

\dedicatory{For/Dedicated to/To my\ldots} 

%----------------------------------------------------------------------------------------
%	THESIS CONTENT - CHAPTERS
%----------------------------------------------------------------------------------------

\mainmatter % Begin numeric (1,2,3...) page numbering

\pagestyle{thesis} % Return the page headers back to the "thesis" style

% Include the chapters of the thesis as separate files from the Chapters folder
% Uncomment the lines as you write the chapters

% Chapter Template

\chapter{Introduction} % Main chapter title

\label{Chapter1} % Change X to a consecutive number; for referencing this chapter elsewhere, use \ref{ChapterX}

%----------------------------------------------------------------------------------------
%	SECTION 1
%----------------------------------------------------------------------------------------
Taking a photograph of an object, traditionally, we need to face a camara (detector) to the object. But with two-photon imaging 
we use a detector that is towards the light source, rather than towards the object.
As the name suggest it, we also use the information about another photon that is strongly correlated. (IMAGE)
 Two-photon is reproduced at quantum level by a non-factorizable point-to-point image-forming correlation between two photons.

Two-photon imaging has been demonstrated using two types of light sources. Type-one
two-photon imaging uses entangled photon pairs as the light source. In 1995 Pittman, realized a 
quantum two-photon geometric optical effect.  They have successfully performed optical 
imaging by means of a quantum-mechanical entangled source\cite{pittman}.

Type-two of imaging uses chaotic light. The type-two  
image-forming correlation is caused by the superposition between paired two-photon amplitudes,
or the symmetrized effective two-photon wave-function\cite{physicsGhost}.
\section{Imaging}

Assuming we have an object that have its own light or its externally illuminated,
imaging means collecting that light that is emitted from the object. Each point
of the surface od the object will emit spherical waves to all possible directions,
been this said, What is the probability to have a spherical wave collapsing into a point or small spot? 
Obviously, the chance is practically zero unless an imaging system is applied.
\\
The concept of optical imaging was well developed in classical optics and the Figure
\ref{fig:imaging} schematically illustrates a standar imaging setup. In this setup 
an object is illuminated by a radiation source, an imaging lens is used 
to focus the scattered and reflected light from the object onto an image plane 
which is defined by the “Gaussian thin lens equation”\cite{hecht}:
\begin{equation}
\frac{1}{s_0}+\frac{1}{s_i}=\frac{1}{f}
\end{equation}
 where $s_0$ is the distance between the object and the imaging lens, $s_i$ the distance 
between the imaging lens and the image plane, and $f$ the focal lenght of the imaging lens. This equation defines
a point-to-point relationship between the object plane and the image plane: any radiation starting from a point on the object will colapse at a certain point at the image plane.
\\
\begin{figure}[h!]
\centering
\includegraphics[width=0.6\textwidth]{Figures/imaging.png}
\caption{Optical imaging: a lens produces an image on an object at $f+d$. This distance is defined
by the Gaussian thin-lens equation $\frac{1}{l}+\frac{1}{f+d}=\frac{1}{f}$} 
\label{fig:imaging}
\end{figure}
This one-to-one correspondence in the image-forming relationship between the object and the image planes produces a perfect image.
The observed image can be magnified or demagnified, for example, in the 
Figure \ref{fig:imaging} the original object is a tree, and it is demagnified at the image plane. This depends on which optical 
system are we using, what kind on lenses are involved and the distance between object and them.
The observed image is a reproduction of the illiminated object, mathematically
corresponding to a convolution between the object distribution fuction $ |T(\vec{\rho_o})|^2$ (aperture function) and a $\delta$-function, which is present for the perfect
point-to-point correspondence\cite{introquantum}:
\begin{equation}
I(\vec{\rho_i})=\int_{obj} d\vec{\rho_o} |T(\vec{\rho_o})|^2 \delta(\vec{\rho_o}+\frac{\vec{\rho_i}}{m})
\end{equation}
where $I(\vec{\rho_i})$ is the intensity at the image plane, $\vec{\rho_o}$ and $\vec{\rho_i}$ are 2-D vectors of the
transverse coordinates in the object and image planes, respectively, and
$m=s_i/s_o$ is the image magnification factor.

In reality, we are limited by the finite size of the optical system, we may never obtain a perfect image.
we have to take into account the constructive-destructive interference present
in this phenomena, because of the wave nature of light. The point-to-point correspondence turns into a point-to-"spot" relationship.
For further informations about this "real life" situation check the \ref{app:intensity}.



%-----------------------------------
%	SUBSECTION 1
%-----------------------------------
\section{Two-Photon Imaging}

The optical imaging used the photons at the image plane to form the image, in other 
words it take measure one photon per spot at the image plane. For the type-one and type-two
two-photon imaging, in certain aspects the behaviour is similar as that of the classical.
They both exhibit a similar point-to-point imaging-forming function, except the 
two-photon image is only reproducible in the joint-detection between two independent photodetectors,
and the point-to-point imaging-forming function is in the form of second-order correlation,
\begin{equation}
R_{12}(\vec{\rho_i})=\int_{obj} d\vec{\rho_o} |T(\vec{\rho_o})|^2 G^{(2)}(\vec{\rho_o},\vec{\rho_i})
\end{equation}
where $R_{12}(\vec{\rho_i})$ is the joint-detection counting rate between photodetectors $D_1$ and $D_2$.
$G^{(2)}(\vec{\rho_o},\vec{\rho_i})$ is a nontrivial point-to-point second-order correlation
function, corresponding to the probability of observing a joint photo-detection event
at the coordinates $\vec{\rho_o}$ and $\vec{\rho_i}$. The physics behing $G^{(2)}(\vec{\rho_o},\vec{\rho_i})$
is what changes between type-one and type-two two-photon imaging.
\begin{figure}[h]
\centering
\includegraphics[width=0.5\textwidth]{Figures/pittman.png}
\caption{Schematic of the first "two-photon imaging" experimental setup, used by Pittman\cite{pittman}} 
\label{fig:pittman}
\end{figure}

\subsection{Two-Photon Imaging using entangled photon pairs}
The first type-one two-photon imaging experiment was demonstrated by Pittman
in 1995\cite{pittman}. The schematic setup of the experiment is shown in the 
Figure \ref{fig:pittman}. A continuous wave (CW) laser is used to pump a nonlinear 
crystal to produce pairs of entangled photons. 
This pairs of orthogonally polarized signal and idler photons are the product
of the nonlinear optical process of spontaneous parametric down-conversion (SPDC).
The pair emerges from the crystal collinearly, it is separated by a dispersion prism, 
and then the signal and idler are sent in different directions by a polarization
beam slitting Glan-Thompson prism. 

The experimental setup is shown in Fig. 1.A 2-mm-diam beam from the 351.1-nm 
line of an argon ion laser is used to pump a nonlinear beta barium borate (BBO)
(P-BaBz04) crystal that is cut at a degenerate type-II phase-matching angle to produce 
pairs of orthogonally polarized signal (e-ray plane of the BBO) and idler (o-ray plane 
of the BBO) photons. The pairs emerge from the crystal nearly col- linearly, 
with ~,=~;=cup/2. The pump is then separated from the slowly expanding 
down-conversion beam by a UV grade fused silica dispersion prism and the 
remaining signal and idler beams are sent in different directions by a polarization 
beam-splitting Thompson prism. The refIected signal beam passes through a 
convex lens with a 400-mm focal length and illuminates the (UMBC) aperture. 
Behind the ap- erture is the detector package $D_1$, which consists of a 25-mm 
focal length collection lens in whose focal spot is a 0.8-mm-diam dry ice 
cooled avalanche photodiode. The transmitted idler beam is met by detector 
package $D_2$, which consists of a 0.5-mm-diam multimode fiber whose output is 
mated with another dry ice cooled avalanche photodiode. Both detectors are 
preceded by 83-nm-bandwidth spectral filters centered at the degenerate 
wavelength 702.2 nm. The input tip of the fiber is scanned in the transverse 
plane by two orthogonal encoder drivers, and the output pulses of each 
detector, which are operating in the Geiger mode, are sent to a coincidence 
counting circuit with a 1.8-ns accep- tance window.


An important fact of this experiment is the use of a lens(collection lens) in the signal beam that establishes an image plane with the definitive point-by-point correspondence object(mask) plane.




\subsection{Two-photon Imaging Using Thermal Sources}

In \cite{thermal} they compared ghost Imaging using entanglement versus Classical correlated light. read and information in \cite{thermalAlejandra}. here i talk about coherence and intensity fluctuations \cite{intensity}

\subsection{Simulations}

No\cite{simulated}
%----------------------------------------------------------------------------------------
%	SECTION 2
%----------------------------------------------------------------------------------------




% Chapter Template

\chapter{Theoretical Discussion} % Main chapter title

\label{Chapter2} % Change X to a consecutive number; for referencing this chapter elsewhere, use \ref{ChapterX}



 \section{Imaging}

\section{Two-photon Imaging}


%----------------------------------------------------------------------------------------
%	SECTION 1
%----------------------------------------------------------------------------------------

\section{Light Source}

main source of information \cite{physicsGhost}
%-----------------------------------
%	SUBSECTION 1
%-----------------------------------
\subsection{Biphoton}

\begin{equation}
\ket{\Psi}=\int dq_s dq_i d\Omega_s d\Omega_i 
x [\Phi(q_s,\Omega_s;q_i,\Omega_i) \hat{a}^{\dagger} (\Omega_s,q_s) \hat{a}^{\dagger}(\Omega_i,q_i) \\
+ \Phi(q_i,\Omega_i;q_s,\Omega_s) \hat{a}^{\dagger}(\Omega_s,q_s) \hat{a}^{\dagger}(\Omega_i,q_i)]   \ket{0}  
\end{equation}

taken like it appears on \cite{spatiocorrelations}

 Where $\Phi(q_s,\Omega_s;q_i,\Omega_i)$ are the mode fuctions or Biphotons, a fuctions that contain all the information about the correlations. $ \hat{a}^{\dagger}(\Omega_n,q_n)$ the creation of a photon with tranverse momentum $q_n$ and frequency $\Omega_n$

%-----------------------------------
%	SUBSECTION 2
%-----------------------------------

\subsection{Mode Function}


\begin{equation}
\Phi(q_s,\Omega_s;q_i,\Omega_i) \propto E_p(q_p,\Delta_0) B_p(\Omega_p) \mathcal{C}_{spatial}(q_s) \mathcal{C}_{spatial}(q_i) 
 x \mathcal{F}_{frequency}(\Omega_s) \mathcal{F}_{frequency}(\Omega_i) sinc \left( \frac{\Delta_k \mathcal{L}}{2} \right)
\end{equation}
where $B_p(\omega^0_p+\Omega_p)$ and $E_p(q_p)$ are the frequency and transverse momentum distribution of the pump. $\mathcal{C}_{spatial}(q_n)$ spatial filtering. $\mathcal{F}_{frequency}(\Omega_n)$ frequency filter function.


\subsection{SPDC}
\cite{spatiocorrelations}

\subsection{Phase matching conditions}
\begin{equation}
\Delta_0 = q^y_s cos \varphi_s + q^y_i cos \varphi_i + k_s sin \varphi_s - k_i sin \varphi_i ;
\end{equation}
\begin{equation}
\Delta_k = k_p - k_s cos \varphi_s - k_i cos \varphi_i - q^y_s sin \varphi_s + q^y_i sin \varphi_i 
+ (q^x_s + q^x_i ) tan \rho_0 cos \alpha + \Delta_0 tan \rho_0 sin \alpha 
\end{equation}
where $k_n=[(\omega^0_n n_n / c )^2 - |q_n|^2]^{\frac{1}{2}}$ is the longitudinal wavevector inside the crystal. $\varphi_s$ and $\varphi_i$ are the propagation directions of the generated photons inside the crystal with respect to the pump direction $z$ and $\alpha$ is the azimuthal angle.

\subsection{Gaussian approximations}

\cite{spatiocorrelations}

Taking into account the Gaussian nature of the pump, that's $E_p(q^x_p , q^y_p ) \approx exp \left[ -\frac{w_p^2}{4}(q^{x^2}_p + q^{y^2}_p )\right]$.

approximating the sinc function by a Gaussian function with the same width at $\frac{1}{e^2}$ of its maximum, i.e., $sinc(x)\approx exp(- \gamma x^2)$  with $\gamma$ equal 0.193. 

\begin{equation}
\mathcal{F}_{frequency}(\Omega_n) \approx exp \left[-\frac{ \Omega^2_n}{4 \sigma^2_n} \right] 
\end{equation}
\begin{equation}
\tilde{\Phi}(q_s,q_i)=\int d\Omega_s d\Omega_i \mathcal{F}_s (\Omega_s) \mathcal{F}_i (\Omega_i) \Phi(q_s,\Omega_s;q_i,\Omega_i)
\end{equation}

The Biphoton then takes a quadratic form:
\begin{equation}\label{eq:quadratic}
\tilde{\Phi}(q_s,q_i)=N exp\left[ -\frac{1}{2}x^T A x + i b^T x \right]
\end{equation}
where N is a normalization constant, $x$ is a 4-dimensional vector defined as $x = (q^x_s, q^y_s ,q^x_i,q^y_i )$, $A$ is a 4 x 4 real-valued, symmetric, positive definite matrix and b is a 4- dimensional vector. A and b are defined from the phase-matching conditions of the SPDC process. $x^T$ and $b^T$ denote the transpose of $x$ and $b$. $A$ and $b$ are functions that depend of all the relevant parameters in the experiment such as the length of the crystal L, pump waist $w_p$, creation angles inside the crystal $\varphi_n$ and the width of the spectral filter $\sigma_n$.

A way to quantify the degree of spatial correlation we shall define 'correlation parameter':
\begin{equation}
K^\lambda = \frac{C^\lambda_{si}}{\sqrt{C^\lambda_{ss}C^\lambda_{ii}}}
\end{equation}
calculated for each direction $(\lambda = x, y)$ from the covariance matrix $C^\lambda$ with elements $C^\lambda_{kj} = \langle q^\lambda_k q^\lambda_j \rangle - \langle q^\lambda_k \rangle \langle q^\lambda_j \rangle $.


\subsection{Fresnel Propagator}


Fresnel Propagator: $h(r,z)=(- \frac{i}{\lambda z})e^{(i \frac{2 \pi z}{\lambda})} \Psi (r,z)$ 
with $\Psi(r,z) = e^{(i \frac{\pi}{\lambda z })r^2}$. Thin-lens transfer function $L_f (r)=\Psi(r,-f)$
 \\
\begin{equation}\label{eq:green}
G= \int d^2 r_1 \int d^2 r_0 h(r_f - r_1,f) L_f(r_1) h(r_1 - r_0,f)
\end{equation}

The propagation is done by determining the Green function\cite{green} of the optical path
by which the beam will travel. The biphoton function
in terms of transverse momenta $\Phi_1 (q_s , q_i )$ after traveling
through two arbitrary optical paths can be expressed
in terms of the corresponding Green functions and the
initial biphoton function $\Phi(q_s , q_i )$ as:

\begin{equation}
\Phi_1 (q_s , q_i )= G_s(q_s,r_1) G_i(q_i,r_2) \Phi (q_s,q_i) 
\end{equation}
\begin{equation}
\Phi_1 (r_1 , r_2 )= \int d^2 q_s d^2 q_i \Phi_1 (q_s , q_i ) 
\end{equation}

Taking advantage of the 2-F system as a Fourier-Transform to reduce the amount of calculations. Solving \ref{eq:green} over $r_0$ and $r_1$ we have:
\begin{equation}
G(q,r_f)=C e^{\frac{i \pi}{\lambda f} r_f^2} e^{\frac{i \lambda f}{4 \pi} q^2} \delta ( q - \frac{2 \pi}{\lambda f}r_f)
\end{equation}
where C is a complex constant that depends only on $\lambda = 2\pi c$ and $f$. Then we can define the Green Functions for each path:

\begin{equation}
G_1(q_s,r_1)=G(q_s,r_1) x T(r_1) 
\end{equation}
\begin{equation}
G_2(q_i,r_2)=G(q_i,r_2)
\end{equation}

Where $T(r_1)$ is the transfer function of the object.

Gathering all the previous results we can obtain $\Phi_1 (r_1 , r_2 )=C^2 T(r_1) \Phi (\frac{2 \pi}{\lambda f}r_1, \frac{2 \pi}{\lambda f}r_2)$, which describes the biphoton at the planes of the object and the scanning detector. It shows that the biphoton at the 2F plane in terms of
$r_1$ and $r_2$  has the same form as the biphoton at the
output face of the crystal with the relationship $q = \frac{2 \pi}{\lambda f} r$.
This allows to computationally simulate the biphoton at the 2-F plane by using Eq \ref{eq:quadratic} without the need to computationally simulate its propagation through the 2-F system.

We are collecting all the light that interacts with the object by the means of a bucket detector, this from the mathematical point of view leave us with: 
 $\Phi_1 (r_2) = C^2 \int d^2 r_1 T(r_1) \Phi (\frac{2 \pi}{\lambda f}r_1, \frac{2 \pi}{\lambda f}r_2)$ 
The coincidence counts that will be measured by the Detectors will be proportional to the magnitude square of the resulting biphoton function $\Phi_1 (r_2)$.
\begin{equation}
S(r_2) \propto |  \int d^2 r_1 T(r_1) \Phi (\frac{2 \pi}{\lambda f}r_1, \frac{2 \pi}{\lambda f}r_2) |^2
\end{equation}

For non-ideal forms of $\Phi (q_s,q_i)$ we have the relation between $\Phi (q) \rightarrow \Phi (r)$ for a 2F system, Hence: $\Phi(r)=\frac{1}{\sqrt{det(\Sigma)(2 \pi)^4}} e^{- \frac{1}{2} r^T \Sigma^{-1} r} e^{ibr}$ \\
$\Sigma=$


--------------------------------------------------
\section{Fourier Optics}
\section{Spatial Correlations}
\cite{omar} whenever i talk of spatial correlations and the especific case

\section{Measurement}
Paper about stadistics of the messuarement\cite{opticalComunications} 
 
% Chapter Template

\chapter{Experimental Setup} % Main chapter title

\label{Chapter3} % Change X to a consecutive number; for referencing this chapter elsewhere, use \ref{ChapterX}
In the following chapter, we will look in detail the components of the experimental 
setup to observe the effects of different spatial correlations in a lensless quantum image
experiment. The setup consist of two  main stages. The first one is the light source 
with tunable spatial correlations, and the second one the two-photon imaging system. 
This experimental setup located at the optical table of the 
Quantum Optics Laboratory.
%----------------------------------------------------------------------------------------
%	SECTION 1
%----------------------------------------------------------------------------------------
\section{Light Source with Tunable Spatial Correlations}

The experimental setup to obtain pairs of photons is shown in Figure \ref{fig:SPDC}. 
The source consists of a type-II crystal in  non-collinear configuration. The source is based
on spontaneous parametric down conversion.


\begin{figure}[h!]
\centering
\includegraphics[width=0.8\textwidth]{Figures/SPDC.png}
\caption{Experimental Setup for the SPDC light Source} 
\label{fig:SPDC}
\end{figure}


To obtain pairs of photons by means of SPDC, we need to pump a nonlinear crystal.
For this experiment, we use a diode laser (Crystal Laser model No. DL 405-200),  
that delivers a continuous wave(CW) at wavelength $\lambda = 406,101 nm$ and $\Delta \lambda = 4 nm$. 
The laser delivers light at 200 mW with a beam diameter 
of 1.5 mm and a beam Divergence of 1.2 mrad.






As seen in Figure \ref{fig:SPDC}, we redirect the laser beam two times, for doing
so we use a pair of mirrors. For this kind of experiments, when the efficiency 
of the optical elements is really important, it is important to use the correct type
of mirror, we want a mirror that reflects most of the light. For this reason, depending
on the wavelength it is posible to find mirrors and lenses with different types of 
coating. Mirror and Lenses have a thin layer that is more efective for a range 
of wave lengths. For our experiment the mirrors have a coating that highly reflects light at
$405nm$. It is posible to manipulate the direction in which the mirror will reflect the 
light by using an appropriate mount with screws that allows to move the reflected beam in one
direction. In the experimental setup we use two mirrors to change the direction two times.



\subsection{Spatial Filter to achieve a Gaussian pump beam}
As seen in theory we need a gaussian beam to pump the crystal. In our experiment we have a 
diode laser whose spatial profile is not a Gaussian. This ramdom spatial profile is a result of the randomnes in the quantum emissions and 
absorptions that are happening at the exited atoms at the diode laser\cite{hecht}.
In order to achieve a Gaussian profile, we use the spatial filter presented in the 
Figure \ref{fig:SPDC}. The spatial filter is composed by two irises, an Aspheric Lens of $f=30 mm$(LA1805-A), a pinhole of $50 \mu m$ and a 
collimating lens of $f=60 mm$(LA1134-A).
\begin{figure}[h!]
\centering
{  \includegraphics[width=0.6\textwidth]{Figures/inputBeam.png} }
{  \includegraphics[width=0.6\textwidth]{Figures/outputBeam.png} }
\caption{The spatial intensity profile before and after the spatial filtering process , Taken from \cite{thorlabs}}
 \label{fig:inputOutputBeam}
\end{figure}

To undestand what a spatial filter does consider a light beam with a spatial profile as the 
one depicted in Figure \ref{fig:inputOutputBeam}(a). After passing a spatial filter the 
obtained light beam look like the one in Figure \ref{fig:inputOutputBeam}(b), that is 
now the Gaussian profile we were looking for.

In Figure \ref{fig:int} we can appreciate the spatial intensity profile of the pump beam just after 
it goes through the spatial filter, the red line is a fit done computationally. The green dots are the measured waist at 3 given
percentages of intensity, $13.5\%$, $50\%$ and $80\%$ respectably. The tool we used to measure this is called Beam Master, and it have its own spatial directions, V and W direction . 
In the optical table we made sure that the V and W direction agreed with our x and y directions. The waist of a gaussian beam propagating in the z direction is given by\cite{waist}:
\begin{equation}\label{eq:wa}
w(z)=w_0 \sqrt{1+(z/z_R)^2}.
\end{equation}
Where $z_R$ is the \textit{Rayleigh length} and it's defined as $z_R=\frac{\pi n w_0^2}{\lambda_0}$. The previous expresion is a function 
of $n$ the refractive index, $\lambda_0$ the wavelength of the beam and $w_0=w(0)$ the waist at the origin. 
 


\begin{figure}[h!]
\centering
{  \includegraphics[width=0.48\textwidth]{Figures/inteV.png} }
{  \includegraphics[width=0.48\textwidth]{Figures/inteW.png} }
\caption{The spatial intensity profile measured after the spatial filter}
 \label{fig:int}
\end{figure}



\subsection{Lenses to Control the pump beam waist}

After successfully achieving a Gaussian profile, which is important for the reasons 
described in Section \ref{sec:spatialCorrelations}, we need a way to control the pump waist, 
but we need to do it 
in a way not too complicated, that doesn't imply too many changes in the experimental setup.
 Putting a  lens, in the beam propagation direction, with certain focal lenght $f$ will define
 a zone around the distance $f$ called \textit{Focus depth}\cite{hecht}, where in the middle 
we find the narrowest point of the beam. The radius of this zone is:
\begin{equation}
 W_0=\frac{\lambda f}{\pi W_B}.
\end{equation}
Where $W_B$ is the width of beam at the lens. In Figure \ref{fig:waist} we see this 
behaviour.  If we want to change $W_0$, we need to use a different lens with $f'$ a different
focal length. This will produce also a different \textit{Focus depth}. Therefore, 
if we want to focus the beam at a fixed distance $F$, using this method to control the pump
 waist is not practical. Every different lens we would use will make this waist $W_0$ at a
 differents distances $f$. It is necessary to find a combination of lenses, that we will call
\textit{waist lens} that make us a 
$W_0$ at a transverse plane located in a fixed position $F$ from the set. 
In Figure \ref{fig:fixed} we present the combination of lenses, \textit{waist lens}.
 It consists on an arrangment of two 
lenses, a positive and a negative one, separated a distance $d_0$ from each other.
\begin{figure}[h!]
\centering
\includegraphics[width=0.5\textwidth]{Figures/waist.png}
\caption{Effect of lens on a Gaussian beam\cite{hecht}} 
\label{fig:waist}
\end{figure}


\begin{figure}[h!]
\centering
 \includegraphics[width=0.65\textwidth]{Figures/fixed.png}
\caption{Composition of lenses to control the Pump Waist at a Fixed distance $F$} 
\label{fig:fixed}
\end{figure}
We can define a \textit{effective focal length} $F$ as a function of  $f_1$ and $f_2$, the focal lengths of the positive and negative lenses 
respectively. With the constrain that $d_0 < f_1$, $F$ reads:
\begin{equation}
F=\frac{f_1 |f_2|}{|f_2|-f_1+d_0}.
\end{equation}
This new effective focal length is crucial in the realisation of this experiment, as described before, we are interested in observing the effect 
in the reconstructed image using different spatial correlations. This is done by changing the pump waist of the laser that is focused on the crystal.
It is not experimentally practical to be changing the crystal position, it would mean to change the position of all the optical elements that 
follows. For this reason, it is perfect to be able to achieve the desired pump waist just by changing the relative separation of two lenses. 


\subsection{BBO(Beta Barium Borate) Crystal: The source of pair of photons}
The Beta Barium Borate (BBO) is an inorganic compound, with chemical formula $\beta$-BaB$_{2}$O$_4$. This crystal is a
nonlinear optical media commonly used. It is also a birefringent\footnote{Birefringence is a optical property of some materials of 
having a refractive index that depends on the polarisation and the propagation of light\cite{hecht}} material and its transmission regions extends
from $189nm$ to $3500nm$\cite{bbo}.
The type-II crystal is mounted is such way that the input and output plane are fixed. In particular the input plane is at $F$ from 
the \textit{waist lens} presented in the previous section, the power of the pump at this point is $\sim 60mW$. The generated photons doubles the wavelength 
of the original pump, it means the generated photon are around $810nm$.


\begin{comment}
\begin{figure}[h!]
\centering
{\includegraphics[width=0.35\textwidth]{Figures/bbo.jpg}}
{  \includegraphics[width=0.5\textwidth]{Figures/aPhotonBPhoton.png} }
\caption{(a):Actual BBO crystal used in experiment. (b): the noncolinear configuration presented in this experiment}
 \label{fig:bbo}
\end{figure} 
\end{comment}

At this point we have as a result of the SPDC process a pair of entangled photons, which have an strong correlation. This correlation is 
the feature in which we are interested on. We need to observe the shape of this correlations functions and the next section will focus 
on the experimental setup that will allow us to observe this.



\section{Spatial Correlations Measurement Setup}
From this point we will talk about a pair of correlated photons, that will come from the output plane of the BBO 
crystal, for historical reasons this photons are labeled as \textit{signal} and \textit{idler}. Nevertheless, to keep the same notations used through 
this monograph, this photon are going to be labeled as A-photon and B-photon, depending on which path they follow, in Figure \ref{fig:spatialSetup}
we can see the experimental setup for measuring the spatial correlations.

\begin{figure}[h!]
\centering
\includegraphics[width=1\textwidth]{Figures/spatialCorrelationSetup.png}
\caption{Experimental Setup for Obtaining the spatial correlations of a pair of down-converted photons} 
\label{fig:spatialSetup}
\end{figure}

Observing the theory developed before there is a relation between the transverse momentum $\vec{q}_n$ of the light before the 2-f system, and the photon 
position $\vec{r}_n$ after de system, Eq. \ref{eq:fourier}. For this reason we placed two lenses, one in each path of the light.
Doing this will put our detections in the Fourier plane. We use two lenses(LA1708) of $f=200.0mm$ in front 
of each \textit{A} and \textit{B}. It is also important to note that from now on, the lenses and mirror used from here, will have a coating that
transmits light around $810nm$ with high efficiency for the lenses, and highly reflects light at this wavelength. 




We are interested in just a pair photons, $\Phi(\vec{q}_B,\Omega_B;\vec{q}_A,\Omega_A)$, that are polarised in certain direction. In order to filter the others
photons $\Phi(\vec{q}_A,\Omega_A;\vec{q}_B,\Omega_B)$, and obtain the Eq. \ref{eq:stateFun}, we place a pair of polarisers at both paths. A polariser is an optical element that filters light
depending on the direction of the electrical field. We used a pair of Polarisers(WP25M-UB), which consist of an array of parallel metallic
wires sandwiched between glass with certain coating for better transmission.





As pointed out in Section \ref{sec:spatialCorrelations}, in order to observe the transverse
correlations, the frequency information has to be traced out. For doing so, we placed a 
pair of Interferometer Filters(IF), Thorlabs FB810-10. They are modeled as $f_n (\Omega_n)=\text{exp}[-\Omega_n^2/(
4\sigma_n^2)]$, where for this specific case $\Omega_n=810nm$ and $\sigma_n=2nm$. This optical elements 
have the special feature that only transmits light that comes throught this range of frequencies. 

\subsection{Detection Module}
To observe the spatial correlations we have to be able to measure light that is propagating
in the z-direction. Figure \ref{fig:scan} shows the plane that is being scanned, where each 
square have a $x_i$ and $y_j$ position, ${i,j}$ goes from $0$ to $N$. With the help of a motorised translational stages, we can make this $NxN$ steps. We can 
control the movement of a pin hole detector, which consists in a single mode optical fiber tip. The translational stages are controlled 
by Arduinos, this enable us to do the scan in a complete automated way.
\begin{figure}[h!]
\centering
\includegraphics[width=1\textwidth]{Figures/scan.pdf}
\caption{The plane that is being scanned by the fiber tip, it is a $4x4mm$ square, that
can be scanned in N steps, where N is defined by us} 
\label{fig:scan}
\end{figure}

Another feature that is easily controlled, is the exposure time. It means we can set how many seconds, is going to be the fiber tip at
every ($i,j$) position. A greater time means more photon counted, and with a bigger amount of data of photons
counted per position, the means values per position gives a better image, with better contrast. The places where we don't have photons 
tend to have a low mean value of photons counted, while the more intense places keep counting, hence having bigger means values.
The ($i_B,j_B$) position and ($i_A,j_A$) position are related with $\vec{q}_B=(q_i,q_j)$ and $\vec{q}_A=(q_i,q_j)$ in Eq \ref{eq:quadratic} 
$\tilde{\Phi}(\vec{q}_B,\vec{q}_A)$, respectively. The spatial correlation we seek to observe. When taking a Two-photon imaging we already deduce in the 
previous Chapter that the image is going to be related with $R(\vec{r}_A)$ from Eq. \ref{eq:R}, where the ($i_A,j_A$)
 position is related with $\vec{r}_A=(x_i,y_j)$.






\subsection{Single Photon Counting Module(SPCM)}

Light is transmitted through an optic fiber from the pin hole detector to the SPCM. This 
consists in a self-contained module that detects single photons of light over the $400nm$ to $1069 nm$
wavelength range. The module used  is SPCM-AQRH-13, and it uses a unique silicon avalanche photodiode (SLiK) with a detection efficiency of more than 65\%\cite{spcm}.
The result signal coming from the SPCM are pulses where each one represents one photon.
\begin{figure}[h]
\centering
\includegraphics[width=0.35\textwidth]{Figures/spcm.png}
\caption{Single Photon Counting Module} 
\label{fig:spcm}
\end{figure}


\subsection{Field-programmable gate array(FPGA)}
Both \textit{A} and \textit{B} pulses from the respective SPCM goes to the same Field-Programmable Gate Array (FPGA). This
FPGA (ZestSC1) is programmed
to count the photon coincidences, this means that the FPGA is fast enough to detect and separate pulses from photons 
that are time-separated. 




\subsection{Computer(Data Analysis)}
LabvVIEW is used to control the detection module, and also, to recibe and translate the information
from the FPGA. It deliver the single and coincidence counts for every position in the 
scan grid, Fig. \ref{fig:scan}. Using this information is only matter of use any way to handle
this data and generate the graph for single and coincidence counts. Through this monograph
it has been used the python language and the matplotlib library to generate them.


\section{Two-Photon Imaging Setup}
Figure \ref{fig:ghostSetup} shows the extra parts of the experimental setup 
for doing Two-photon imaging. In path B we add an object and a bucket detector $D_C$. 
It may be noted that all the setups shown so far, use in essence the same optical elements.
 It is important to create an experimental setup that allows us to 
measure different things without changing it too much. For this we re-direct the light that 
goes through the object by means of a flip mirror (FM).

\begin{figure}[h!]
\centering
\includegraphics[width=1\textwidth]{Figures/ghostSetup2.png}
\caption{Experimental Setup for the Two-photon Imaging} 
\label{fig:ghostSetup}
\end{figure}


The object is an obstruction that is placed in the \textit{B} path. This is the object
from which we will make an image. It consist on an aperture $T(r_B)$ on a translational mount,
that allow us to move the aperture precisely in the same plane we make our detections.
This is done by manipulating a pair of screws. 
We used differents objects and in Figure \ref{fig:mask1}
there is a detailed schematic of the first one used. It consists on a square aperture placed in the 4th 
quadrant of the scanned plane. 

\begin{figure}[h!]
\centering
 \includegraphics[width=0.6\textwidth]{Figures/mask1.pdf}
 \caption{Mask 1: Detailed description of the 'square' aperture location in the scanned plane}
\label{fig:mask1} 
\end{figure}

In Figure \ref{fig:masks} there are the other two apertures that we used so far in the 
experiment, 
these apertures were selected because of the symmetries and antisymmetries they present, 
and therefore
they helps us to recognise the effects  of the spatial correlations in the image
recovered.  

\begin{figure}[h!]
\centering
{  \includegraphics[width=0.45\textwidth]{Figures/mask2.png} }
{  \includegraphics[width=0.45\textwidth]{Figures/mask3.png} }
\caption{Mask 2: The letter 'L' pointing down. Mask 3: Opening question mark.}
 \label{fig:masks}
\end{figure}


In order to change de path followed by the B-photon, and guide the light to a new detector $D_C$, we use a Folding mirror.
This mirror plays the role of a switch, when is up, we are know dealing with the
$D_C$ detector, and we are doing a Two-photon Imaging process. In contrast, when the mirror is 
down, we are recovering spatial information, so it is possible to recognise some sort of
shadow from the object, or to measure the spatial correlations.



The bucket detector $D_C$ consists in a coupling lens, a multimode fiber and a detector. The lens collects all the 
light that goes 
through the object couple it to the fiber that is connected to the SPCM. In contrast to the other detections made 
before with $D_A$ and $D_B$, the Bucket detector loses track of any spatial information of the photons. 
% Chapter Template

\chapter{Results} % Main chapter title

\label{Chapter4} % Change X to a consecutive number; for referencing this chapter elsewhere, use \ref{ChapterX}

%----------------------------------------------------------------------------------------
%	SECTION 1
%----------------------------------------------------------------------------------------
In This chapter we present the experimental data recovered through the differents 
steps described in the Chapter before. Most the following results consists in 
2-D arrays representing a matrix, where in each position a color is painted, depending
on how many photons were detected in single or coincidences counts.
As we have seen, before making a Two-photon image, there are some process 
that have to be made before. The first thing todo is to achieve a Gaussian behavior 
of the original diode laser. For doing do we have to look at the beam propagation after 
the Spatial Filter. 

\section{Achiving a Gaussian Beam}

PYTHON PROGRAM STILL ON PROGRESS, TALK ABOUT M FACTOR

\section{Finding The Correlated Photons}

After obtaining a Gaussian propagation, and achieve a pump waist that no varies to much
while propagates, we focused the laser at the BBO crystal and with the help of the \textit{waist lens}
we set the $w_p=91 \mu m$. Before observe the spatial correlations of the down converted
photons, we make sure we are seing them. Figure \ref{fig:correlatedPhotonSpot} shows 
two different images recovered, where in Fig.\ref{fig:correlatedPhotonSpot}(A) we found
out that the translational mount of the mask was not well placed, it was cutting some of the light.

\begin{figure}[h!]
\centering
{  \includegraphics[width=0.45\textwidth]{Figures/correlatedPhotonSpot1.png} }
{  \includegraphics[width=0.45\textwidth]{Figures/correlatedPhotonSpot2.png} }
\caption{(A) and (B) shows the B-photon of the down converted pair. In (B) we moved away the translational mount of the mask}
 \label{fig:correlatedPhotonSpot}
\end{figure}

In Figure \ref{fig:correlatedPhotonSpot}(B) we can the B-photon, and now there is
no interference by the translational mount. As said before we placed a pair of polarisers
in order to filter them. In the figure we can see that just one direction of the 
ring in \ref{fig:bbo}(b) while the other is partially filtered.


\section{Experimental Correlations }

Afterwars we would like to observe the shape of the spatial correlations the pair of down converted
photons and see the experimental behaviour of $\tilde{\Phi}(\vec{q}_B,\vec{q}_A)$.
When remembering the definition of $\vec{q}_n$, it is a 2-D vector, containing the information
of the photon in x and y direction. Since we have two photons, each one with two spatial 
variables. Resulting that we can 4 differents correlations for a pair of photons. There
is important to point out that this transverse momentum $\vec{q}_n$ is related with
his equivalent $\vec{r}_n$ the position of the photon, with n making reference to the A and B paths.

In Figure \ref{fig:expCorrelations} there are the correlations in the xx and yy  direction.
The 2-D matrix in Figure \ref{fig:expCorrelations}(\textit{XX Correlation}) is
 the result of repeating this recipe: placing the $D_A$ at a fixed position and scanning the $D_B$ just in the 
x direction, next we move de $D_A$ one position in the x direction. Repeating this N times 
we construct an image of the coincidence counts between $D_A$ and $D_B$ in every position.

\begin{figure}[h!]
\centering
{  \includegraphics[width=0.45\textwidth]{Figures/xxCorrelation.png} }
{  \includegraphics[width=0.45\textwidth]{Figures/yyCorrelation.png} }
\caption{Experimental Spatial correlations between a pair of down-converted photons. \textit{XX Correlation} shows the correlation in the x variables. \textit{YY Correlation} shows the correlation in the y variables, Beam propagating in the z direction. $w_p = 91\mu m$}
 \label{fig:expCorrelations}
\end{figure}

Figure \ref{fig:expCorrelations}(\textit{YY Correlation}) show the spatial correlation in 
the yy direction, this image is done by repeating the same recipe as before, but this 
time scanning and moving in the y direction.
The spatial correlations in this case present a negative behaviour in both XX and YY direction, an anticorrelation.
Meaning that is expected to measure a photon at a negative position at the B 
path if we measured a photon at a positive position at the A path. They both exhibit a elliptical
shape, but the YY correlation is a narrower one, meaning there is a stronger relation
between the pair of photons in the Y direction.

\section{Mask Alignment}

Before making a Two-photon image we need to know that we have placed the mask in the correct 
spot. This correct spot is defined by the Figures \ref{fig:mask1} and \ref{fig:masks}. Which
localisation was decided in function of where the flux of correlated photon was greater.
The following images where produced as the standar image is done. It means we show the shadow
of the aperture in the B path. Every position of the images is the single 
counts of the $D_B$ in the exposure time. It is like the standar image in the sense 
that we are using the spatial information of the light that interacted with the mask.

\subsection{Mask 1}
The Following Figure \ref{fig:localizationSq} shows the final localisation of the first
mask used. It was the final position because its position is really similar to the one described
in Figure \ref{fig:mask1}. For making this image we set the step length to be $0.2mm$ and
the exposure time was 1 second per position.

\begin{figure}[h!]
\centering
\includegraphics[width=0.6\textwidth]{Figures/localizationSq.png} 
\caption{Localization of the mask with an square}
\label{fig:localizationSq}
\end{figure}

\subsection{Mask 2}
Figure \ref{fig:localizationL} shows the localisation of the second mask. While in Figure \ref{fig:localizationL}(A)
there is the initial position of the mask, Figure \ref{fig:localizationL}(B) shows
the new localisation of the aperture after a translation in the y direction. This images
where done by setting the steps to $0.2mm$ and the exposure time to 1 second per position.


\begin{figure}[h!]
\centering
{  \includegraphics[width=0.45\textwidth]{Figures/localizationL1.png} }
{  \includegraphics[width=0.45\textwidth]{Figures/localizationL2.png} }
\caption{Moving the L Mask in order to put it in the most central spot}
 \label{fig:localizationL}
\end{figure}
In Figure \ref{fig:localizationDef} is presented the definitive position of the aperture
before making a Two-photon imaging. If we take a closer look to the Figure, we can appreciate
a higer contrast, this is because in this opportunity we set the steps to be $0.1mm$ and 
the exposure time to be 30 seconds per position. This image is the result of measuring for around 
14 hours.
\begin{figure}[h!]
\centering
\includegraphics[width=0.6\textwidth]{Figures/localizationLLong.png} 
\caption{Long exposure of the definitive localization of the mask, in this try we leave the 
detector in each place for 30 seconds, we also make the steps of the detector smaller, $0.1mm$}
\label{fig:localizationDef}
\end{figure}
\subsection{Mask 3}

The definitive localisation of the third mask used is presented in the Figure \ref{fig:localizationInte}, 
where the step was $0.1mm$ and the exposure time was set to 30 seconds per position.

\begin{figure}[h!]
\centering
\includegraphics[width=0.6\textwidth]{Figures/interrogationLocation.png} 
\caption{Interrogation definitive position}
\label{fig:localizationInte}
\end{figure}


\section{Two-Photon Images}

Finally we get to observe the Two-photon images that are the core of this monograh,
It is important to remember the way this images are obtained. The image $R(\vec{r}_A)$ is a 
function of the coincidence counts between $D_C$ and $D_A$. We scan $D_A$, and as a 
result we obtain a 2-D matrix where each ($i,j$) position is the coincidence count.
\subsection{mask1}
Figure \ref{fig:twoPhotonSq} is the Two-photon image of the square aperture. The maximun
coincidence counts changed position in the image, compared to Figure \ref{fig:localizationSq}.
Nevertheless the shape is identificable, still has a shape of square, but its position
is reflected in both x and y direction. 
\begin{figure}[h!]
\centering
\includegraphics[width=0.6\textwidth]{Figures/two-photonImageSq.png} 
\caption{Experimental Two-photon image recovered for the square aperture}
\label{fig:twoPhotonSq}
\end{figure}

\subsection{mask2}
The Two-photon image of the second mask is presented in Figure \ref{fig:twoPhotonL}. 
In this opportunity it is clear that the more complex shape of the aperture is hardly
identifiable. However, there are some other interesting things to note about the image.
As the original aperture, the image is not symmetrical, and it is not pointing to the
original direction the L was in Figure \ref{fig:localizationDef}. This make us to think 
about reflexions in the image respect from the original mask, but in this case is 
not that easy to detect them.
\begin{figure}[h!]
\centering
\includegraphics[width=0.6\textwidth]{Figures/twoPhoL3.png} 
\caption{Two-photon image recovered for the L aperture.}
\label{fig:twoPhotonL}
\end{figure}



\subsection{mask3}
The third Two-photon image is in Figure \ref{fig:twoPhotonInte}. Again the complex original
shape is barely visible in the image obtained, Nevertheless the image have a more rounded
part at the biger Y position, that hint us about a reflexion of the interrogation symbol,
now it is oriented like \textit{'?'}, so it may present a reflexion in both x and y direction. 
\begin{figure}[h!]
\centering
\includegraphics[width=0.6\textwidth]{Figures/twoPhotonInte.png} 
\caption{Two-Photon Image Interrogation}
\label{fig:twoPhotonInte}
\end{figure}

 
% Chapter Template

\chapter{Discussions and Conclusion} % Main chapter title

\label{Chapter5} % Change X to a consecutive number; for referencing this chapter elsewhere, use \ref{ChapterX}

%----------------------------------------------------------------------------------------
%	SECTION 1
%----------------------------------------------------------------------------------------

We used a light source that generates pair of entangled photons, however, we broke this entanglement and traced out all the temporal
information. Still we were able to recover some of the characteristics of the objects used, this is an evidence of the capabilities of this source light.
SPDC is an simple experimental process with not too many technical problems that is capable of producing pair of photons
that can exhibit a strong correlation.

We presented an alternative techniche to the problem presented at the introduction of this monograh. If we lose the ability to recover
the spatial information of the reflected or scattered light from an object, we described along this work why Two-photon
imaging can be an alternative. We can light the object with an strongly correlated light, and mesure coincidences at a separated detector, and 
we may recover the image.

The Imaging process shared characteristics in both standar and Two-photon imaging vertions. As discussed through this work and in Appendix \ref{AppendixThermal},
The image recovered in coincidence counts, is a convolution between the transfer function $T(\vec{r}_o)$ of the object, 
and a Second order correlations function $G^{(2)}(\vec{r}_B,\vec{r}_A)$. Meanwhile, in the standar imaging process we find out
that the image recovered is a convolution between $T(\vec{r}_o)$ and Sombrero-like function, which acts like 
the 'correlation' function between the light at the object and at the image.




The results presented so far are not completely exempt, the numerical simulations and the experimental data have points in common. 
In the numeral simulations the images suffered reflections about the axis in wich the correlations was negative. In the experimental Two-photon
imaging of the Mask 1, Fig. \ref{fig:twoPhotonSq}, the image was flipped in both directions, this was the expected behaviour that is present
in the numerical solution.

It is clear that in order to achieve a higher spatial resolution we need to move to a regime where the spatial correlations are
stronger, meaning that they are elipsis more narrower, similar to straight lines. Wanting this narrow spatial correlations is equivalent 
to have a Sombrero-like function with a narrower point-spread function in the frame of standar imaging.

This Experimental procedure described through this monograph is the sum of many previous work, all the things done before
were necessary to get to this point were the experiment is. It is a shame the time stipulated for a monograph is so short,
the results presented so far, are just the first ones. To make an complete study of the 
phenomena of Two-photon using tuneable Spatial Correlations it is necessary to change the pump waist, in order to change the shape of 
spatial correlations.


1. signo correlacion relation with the position of the image.... how norrow a spatial correlation and the quality of the resulting image.

4. what is the true nature of the ability of recotrsucting an image, why correlation light intensities.\cite{zhong}
5. esta es la suma de varios trabajaos, se presento los resultados, primeros steps para obversavar los efectos en el ghost 



%\include{Chapters/ChapterInfo} 

%----------------------------------------------------------------------------------------
%	THESIS CONTENT - APPENDICES
%----------------------------------------------------------------------------------------

\appendix % Cue to tell LaTeX that the following "chapters" are Appendices

% Include the appendices of the thesis as separate files from the Appendices folder
% Uncomment the lines as you write the Appendices

% Appendix Template

\chapter{Two-photon Imaging Using Chaotic Sources} % Main appendix title

\label{AppendixThermal} % Change X to a consecutive letter; for referencing this appendix elsewhere, use \ref{AppendixX}

%\subsubsection{Two-photon Imaging Using Chaotic Sources}

In principle the term "thermal radiation" should refer only to radiation coming 
from a blackbody in thermal equilibrium at some temperature T. But with this realisation of thermal radiation
we have to face some characteristics of true thermal fields. Thermal radiation is also referred as chaotic light, 
which have extreme short coherence time. This is because a thermal source contains a large number of independent sub-sources,
such as the trillions of atoms or molecules.These atomic transitions that can be identical or different
act like sub-sources, that emit light into independently and randomly. 



\begin{figure}[h!]
\centering
\includegraphics[width=0.6\textwidth]{Figures/thermalSetup.png}
\caption{Experimental setup for the Two-photon imaging using thermal light, taken from \cite{thermalAlejandra}} 
\label{fig:thermalSetup}
\end{figure}
The source light in Figure \ref{fig:thermalSetup} is the one developed by Martinssen and Spiller\cite{intensity}
which is the most commonly used among the pseudothermal fields. A  coherent laser radiation is focused on a rotating ground glass disk, 
the scattered radiation is chaotic with a Gaussian spectrum. After this, a nonpolarizing beam
splitter (BS) splits the radiation in two distinct optical pths, In the reflected arm an object, with 
transmission function $T(r_1)$, is placed ar a distance $d_A$ from the BSand a bucket detector ($D_1$)
is just behind the object. In the transmitted arm an imaging lens, with focal lenght $f$, is placed at a 
distance $d_B$ from the BS, and a multimode optical fiber ($D_2$) scans the transverse plane
at a distance $d'_B$ from the lens. The output pulses from the two single photon counters are sent 
to an electronic coincidence circuit to measure the rate of coincidence counts.

Once again we expect the joint-detection counting rate between photodetectors $D_1$ and $D_2$ to behave
like the one described in Eq. \ref{eq:coincidences}. But thos rate this coincidence counts is governed
by the second-order Glauber correlation function \cite{glauber}:
\begin{equation}
G^{(2)} (\vec{r}_1; \vec{r}_2) \equiv \langle E^{(-)}_1(\vec{r}_1)E^{(-)}_2(\vec{r}_2) \times E^{(+)}_2(\vec{r}_2)E^{(+)}_1(\vec{r}_1) \rangle
\end{equation}
where the $E^{(-)}$ and $E^{(+)}$ are the negative-frequency and the positive-frequency field operators describing the detection events at the
locations $\vec{r}_1$ and $\vec{r}_2$. The transverse second-order correlation correlation function 
for a thermal source is given by \cite{thermalAlejandra}:
\begin{equation}\label{eq:thermal}
G^{(2)}_{\textit{thermal}}(\vec{r}_1; \vec{r}_2) \propto \sum_{\vec{q}} |g_1(\vec{q},\vec{r}_1)|^2
\sum_{\vec{q}'} |g_2(\vec{q}',\vec{r}_2)|^2 + |\sum_{\vec{q}} g_1^*(\vec{q},\vec{r}_1) g_2(\vec{q},\vec{r}_2)|^2
\end{equation}
where $\vec{r}_i$ is the transverse position of the detector $D_i$, $\vec{q}$ and $\vec{q}'$
are the transverse components of the momentum vectors, and $g_i(\vec{q},\vec{r}_i)$ is the Green's function 
associated with the propagations of the field with transverse momentum $\vec{q}$ from the source, 
to the position $\vec{r}_i$ at the detection plane defined by the detector $D_i$. $g_i(\vec{q},\vec{r}_i)$ is defined in a similar way 
as in Eq. \ref{eq:green}.

It is important to note that there are two main differences with respect to the SPDC case: 
First the presence of a background noise (first term of Eq. \ref{eq:thermal}), which does not exist for SPDC. Second,
the possibility of writing the second term of Eq. \ref{eq:thermal} as a product of the first order correlation
functions, $G^{(1)}_{12}G^{(1)}_{21}$, while there is no way to write the biphoton produced by the 
SPDC as a product of other correlations. Also this term $|\sum_{\vec{q}} g_1^*(\vec{q},\vec{r}_1) g_2(\vec{q},\vec{r}_2)|^2$
Is the interference of intensities of a incoherent statistical assemble of randomly distributed photons.


Following the proces done in \cite{thermalAlejandra}, it can be shown that for any values of distances
$d_A$, $d_B$ and $d_{B}'$ which obey the equation:
\begin{equation}
\frac{1}{d_B - d_A} + \frac{1}{d_{B}'} = \frac{1}{f}
\end{equation}
which clearly has the form on a thin-lens equation, defining a point-to-point correspondence between imaging and object plane.
Then Eq. \ref{eq:thermal} can be simplified as:
\begin{equation}
G^{(2)}_{\textit{tot}}(\vec{r}_2) \propto N + | T \left( \frac{d_A - d_B}{d_{B}'} \vec{r}_2 \right) |^2
\end{equation}
where $T ( \frac{d_A - d_B}{d_{B}'} \vec{r}_2 )$ is the object transmission function ($T(\vec{r}_1)$)
reproduced on the $D_2$ plane. Thanks to this result we can conclude that a thermal source allows reproducing
in coincidence measurements the two-photon image of an object, similarly to the SPDC case, except for a 
constant background noise, where $N$ is proportional to it.


It is possible to establish an analogy
between classical optics and entangled two-photon optics:
the two-photon probability amplitude plays in entangled
two-photon processes the same role that the complex amplitude
of the electric field plays in classical optics  \cite{thermalAlejandra}.
%\include{Appendices/AppendixB}
%\include{Appendices/AppendixC}

%----------------------------------------------------------------------------------------
%	BIBLIOGRAPHY
%----------------------------------------------------------------------------------------

\bibliography{Bibliography}
\bibliographystyle{unsrt}
%\bibliographystyle{siam}

%\printbibliography
%\printbibliography[heading=bibintoc]

%----------------------------------------------------------------------------------------

\end{document}  
